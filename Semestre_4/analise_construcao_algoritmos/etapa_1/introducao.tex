Algoritmos como Tecnologia

Já paramos para pensar como o processos de um computador funciona dentro de um certo período de execução de um código
e como podemos otimizá-lo para que seja executado mais rapidamente ou da forma que foi requerido dentro de um projeto,
como ele e com boa usabilidade? Logo pensamos em mídias das quais usamos no dia a dia, como o bom funcionamento de  um
Sistema Operacional, seja Desktop ou Mobile, um software bem otimizado, como jogos e aplicações de trabalho, até mesmo
em algoritmos de redes sociais para o bom funcionamento.

Suponhamos que os computadores fossem infinitamente rápidos e que sua memória fosse infinita, assim como a máquina de Turing,
não existiria nenhuma razão real do porque usar, mas algoritmos se torna necessário para eficiência dos processos. Como não 
existe e muito possivelmente nunca existirá este computador, tempos recursos limitados que necessitam de bons algoritmos para 
sua execução.

Quando entramos nessa parte de limitação do hardware, aplicação de algoritmos se
torna um fator crucial, pois para cada processo e aplicação temos a vantagem de
usar um algoritmo ou outro, visando melhor eficiência. Temos alguns exemplos que
serão abordados no futuro, porém para efeitos de explicação, o primeiro é
conhecido como Insertion Sort, ou ordenação por inserção, que  leva um tempo
maior de execução do que o Merge Sorte, ou ordenação por intercalação, que
posteriormente será explicado pela notação assintótica do porquê disso. Para
este caso, vemos um caso concreto quando comparamos um computador A mais rápido que implementa Insertion Sort, e um
computador B mais lento que implementa Merge Sort, cada um deve ordenar um arranjo de 10 milhões de números. Supondo que
o Computador A executa 10 bilhões de instruções por segundo, enquanto o B apenas 10 milhões de instruções por segundo,
temos que o computador A é 1000 vezes mais rápido que o B, correto, logo ganhará de qualquer modo, né? Pondo dois
programadores para fazerem o algoritmo de ordenação e o primeiro, muito astuto, consegue fazer um algoritmo que exige
tempo de execução de 2nˆ2 enquanto o programador mediano do computador B fez o Merge Sorte e conseguiu um tempo de 50n
log n. Teremos uma comparação do seguintes casos:

Computador A:

2 x (10ˆ7)ˆ2 instruções / 10ˆ10 = 20000 segundos

Computador B:

50 x 10ˆ7 x log10ˆ7 intruções / 10ˆ7 = 1163 segundos

Conclui-se que o programador B, tendo um computador de poder de processamento muito menor que o programador A, com a
escolha do algoritmo de ordenação por intercalação conseguiu um resultado 17 vezes melhor. Agora tenha isso para
programas mais complexos, o diferença seria como compara um abismo com um pequeno buraco na rua que atrapalha o seu diaa
dia no trânsito.
