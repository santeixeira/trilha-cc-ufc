Notação Assintótica e Ordem de Crescimento

A ordem de crescimento define o tempo de execução do Algoritmo, que isso pode nos mostrar a eficiência e a capacidade de comparar a peformance entre alternativas do mesmo.

Uma vez que a entrada de dados se torna grande em um algoritmo, temos o exemplo do Merge Sort ser mais eficiente que o Insertion Sort, onde temos que o O(n lg n) se torna melhor que O(n2). Para estes casos que o input size cresce e fica suficientemente grande, o uso da Notação Assintótica é a melhor alternativa.

Definição: Sejam f(n) e g(n) duas funções de inteiros positivos com domínio em positivos reais, ou naturais. Dizemos que f = O(g) que significa que f não cresce mais que g dentro do intervalo de tempo de execução) se existir uma constante c > 0, tal que f(n) <= g(n).c.
